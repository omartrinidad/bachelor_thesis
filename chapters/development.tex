\chapter{Desarrollo}
En este capítulo se presenta el método empleado para preprocesar los mamogramas
digitales.

\section{Etapa de recolección}
Las mamografías originalmente almacenadas en discos compactos, fueron extraídas
y ... , removimos los datos privados de las imágenes, los únicos campos son los
que se muestran en la ref{table:dicomdata} 

\begin{table}
  \caption{Card Information of Patient 0001} 
  \label{examplecard}
\begin{center}
{\small
    \begin{tabular}{c|c}
    \hline

    {\bf Etiquetas} & 
    {\bf Descripción} \\
    \hline
           (0000, 0001) & Bits alojados \\
    \hline
    \end{tabular}
}
\end{center}
\end{table}

%Las tareas de \textit{scripting} se llevaron a cabo con el lenguaje de
%programación Python y la libreria PyDICOM.

\section{Método}
En esta sección se describe el método empleado para preprocesar las imágenes.

\subsection{Reducción del área de trabajo}
Los mamogramas recolectados tienen un -fondo-


