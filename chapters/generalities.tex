\chapter{Generalidades}
\label{generalidades}

% restart the count with arabic numbers
\pagenumbering{arabic} \setcounter{page}{1}

\section{Antecedentes}

El cáncer de mama es un grave problema de salud pública, de acuerdo con la
Organización Mundial de la Salud en 2008 el cáncer de mama fue el tipo de
cáncer más común entre las mujeres y el que más muertes provocó
\cite{cancerreport}. De acuerdo con el INEGI las estadísticas son similares en
México \cite{mxcancer, inegi}. La detección temprana de este padecimiento es la
mejor forma de reducir el número de decesos, a su vez, el método más comúnmente
empleado para ello es el estudio de mamografías \footnote{Los términos
mamografía, mamograma y mastografía son equivalentes y serán utilizados a lo
largo de este documento indistintamente.}. 

Las mamografías son radiologías de baja intensidad. Las mamografías analógicas
tradicionales, mejor conocidas como \textit{placas} (SFM, \textit{Screen-Film
Mammography}) están dejando de ser usadas para dar paso al uso de mamografías
digitales (FFDM, \textit{Full-Field Digital Mammography}). El uso de estas
últimas representa una mejora en el proceso de adquisición de la imagen, su
almacenamiento y visualización \cite{pisano2000current}.

% tarea repetitiva, mejorar los de 'tarea subjetiva'
% peor aún --> even worse
El estudio de mamogramas es una tarea a cargo de radiólogos, no es una
actividad trivial, factores como la fatiga o el uso de equipos de calidad no
estándar influyen en la precisión del diagnóstico. Dado que es una tarea
subjetiva hay un margen de error en la opinión de los radiólogos, lo que
provoca diagnósticos conocidos como \textit{falsos positivos} que son una falsa
alarma al paciente o aún peor \textit{falsos negativos}, que evitan que el
paciente conozca la gravedad de su caso y se someta a un tratamiento.

Los sistemas de \textit{detección asistida por computadora} (CAD) son
desarrollados con técnicas de reconocimiento de patrones para identificar
regiones sospechosas en una imagen y alertar de ello al radiólogo. Por otro
lado los sistemas de \textit{diagnóstico asistido por computadora} que al igual
que los primeros también son conocidos como CAD \footnote{Para distinguir entre
los sistemas de detección asistida por computadora y los sistemas de
diagnóstico asistido por computadora algunos autores suelen abreviar los
primeros como CADe y los últimos como CADi o CADx.} van más allá y su fin es
emitir un juicio sobre las lesiones, clasificando a estas como malignas o
benignas \cite{castellino2005computer}. 

Los sistemas CAD fueron diseñados para ayudar a los radiólogos en la
interpretación de imágenes médicas. Aunque la precisión de los sistemas CAD no
es perfecta \cite{fenton2007influence} sirven como una segunda opinión para los
radiólogos. El desarrollo de los sistemas CAD ha sido vertiginoso desde los
años 80's \cite{giger2008anniversary}.

% mandar lo de CPDM a pie de página ¿?
La elaboración de sistemas de este tipo es una tarea compleja, el primer reto a
sortear es obtener imágenes de prueba. El acceso a imágenes médicas aun con
propósitos de investigación es complicado debido a cuestiones de privacidad.
Este trabajo aborda la construcción de una colección de mamogramas digitales
preprocesados (y) de dominio público bautizada como \textit{Colección de
Mamogramas Digitales Preprocesados} (CPDM, por sus siglas en inglés:
\textit{Collection of Preprocessed Digital Mammograms}). Los estudios fueron
recolectadas durante cuatro meses en Hospital de Alta Especialidad Dr. Juan
Graham Casasús.

\section{Planteamiento del problema}

La concepción y creación de bases de datos de mamografías no es un tema nuevo
\cite{nishikawa1998mammographic}. Nishiwaka plantea un enunciado interesante:
\textit{En el futuro, las bases de datos estarán constituidas por mamogramas
digitales; actualmente (1998) mamogramas digitalizados están siendo utlizados}.
Lo que esta tesis aborda es la creación de una colección de mamogramas
preprocesados.

El desarrollo de sistemas CAD comprende varias etapas, la primera de ellas es
el \textit{preprocesamiento}. Algunas técnicas de preprocesamiento fueron
abordadas en este trabajo. El preprocesamiento es la etapa previa al
procesamiento de imágenes \textit{per se}, de preparación
\cite{ponraj2011survey}. 

% duda: se pone esto o se pone qué no sé sabíamos que hacer y luego lo
% descubrimos

Las mamografías son imágenes no convencionales.

\section{Objetivos}

\subsection{Objetivo general} 

Crear un banco de datos de dominio público con mamografías digitales
preprocesadas.

\subsection{Objetivos específicos}

\begin{itemize}
    \item Aplicar los algoritmos adecuados para preprocesar los mamogramas de
    tal manera que mejore la apariencia visual de las mastografías, esto es, que
    se resalten las lesiones al ojo humano.
    \item Crear una ficha electrónica para cada caso que contenga datos
    relevantes como el diagnóstico médico.
\end{itemize}

\section{Justificación}

Aunque ya existen varias bases de datos, ninguna de ellas es una colección
mexicana y pocas de ellas son latinoamericanas. La mayoría de las bases de
datos públicas que existen ofrecen mamografías tradicionales
\textit{digitalizadas} con escáners, una colección con imágenes totalmente
digitales supondría mejores resultados en la ejecución de algoritmos.

También se pueden usar las imágenes para enseñanza y entrenamiento de
radiólogos noveles. El proyecto es útil para la comunidad científica dedicada
al procesamiento y análisis de imágenes médicas, más específicamente de
imágenes mamográficas, como ya se comentó queda abierta la posibilidad de
desarrollar un sistema CAD, lo que representa una ventaja potencial para los
radiólogos.

% mejorar la redacción de este párrafo
Por otra parte, hay que señalar que la mayor parte de los sistemas CADs son
comerciales y representan un gasto para los hospitales. México es un país en
vías de desarrollo que adopta el papel de \textit{consumidor} de tecnología, la
comunidad científica trabaja en el desarrollo de algoritmos para la elaboración
de sistemas CADs y otras tecnologías que permitan la reducción de la brecha
tecnológica.

% notar la palabra creemos
Creemos que una colección de mamogramas digitales es realmente útil a la
comunidad
