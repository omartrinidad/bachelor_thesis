\chapter{Generalidades}

\section{Antecedentes}

El cáncer de mama es un grave problema de salud pública, es una de las
principales causas de decesos en la población femenina de Tabasco y México. Una 
forma de aminorar estas muertes es la detección temprana de este padecimiento, a
su vez, el método más comúnmente empleado en su detección es el estudio de
mamografías, tarea a cargo de radiologos; hay un margen de error en la opinión
de los radiologos.

Es posible incrementar el éxito de los diagnósticos mejorando la apariencia
(calidad) de la imagen, 

An important step in building such
systems is the creation of databases of medical images. 

Dos bancos de datos similares al que se menciona en este trabajo son el Digital
Database for Screening Mammograpy (DDSM) de la University of South Florida (USF) 
\cite{} y el banco de datos MIAS MiniMammographic Database (mini-MIAS) \cite{}. Ambos son
ampliamente usados por la comunidad de investigación sobre procesamiento y análisis de
imágenes. El objetivo de este trabajo es crear un banco de datos similar a los que
se mencionan arriba. 

Both of them are widely used by the mammographic image analysis research
community.  Our main goal is to create a database similar to the above.  Our
project was supported by “Dr. Juan Graham Casasus” hospital who gave us a batch
of raw mammograms in DICOM file format [3].  especificar, el problema DEL
hospitlal --diagnóstico-- segunda opinión, decir que no se ha implementado algo
así, pero que existe
 
.... info on what to look at for image treatment 

La introducción de [15] da una explicación buena de por qué son buenos los
sistemas CAD.

\section{Problema}
\subsection{Planteamiento del problema}

It was necessary to discover which filters are the best choice in order to
obtain optimized images ready for afterward phases such as segmentation,
processing or analysis. Namely, it was necessary to reduce the noise and other
artifacts in the image. Furthermore, we need to create an electronic file for
each image. These data are obtained from each DICOM file and with the help of
specialized doctors.  The most important data is the medical mammogram
diagnosis. Mammograms are classified according to the American College of
Radiology (ACR) in BI-RADS. Mammography Atlas, edition 4th [4].

The assessment categories are:
a. Mammographic assessment incomplete:
1. Category 0: Need additional imaging evaluation and/or prior mammograms for comparison.
b. Mammographic assessment complete:
1. Category 1: Negative.
2. Category 2: Benign finding(s).
3. Category 3: Probably benign finding.
4. Category 4: Suspicious abnormality.
5. Category 5: Highly suggestive of malignancy.         
6. Category 6: Proven malignancy.

\subsection{Delimitation of the investigation}
\subsubsection{Scope}

decir que hay un estado inicial y uno final
la extensión del trabajo incluye --- filtros y tipos de filtros aplicados, imágenes listas para etapas posteriores. 
hablar de base de datos más como un dataset que otra cosa --- estructura: imagen cruda, procesada y ficha


a database of preprocessed mammograms, namely, mammograms without noise,
reduction of artifacts and leveling of image quality.

\subsubsection{Limitations}

3. No se utilizarán clasificadores para evaluar el resultado, la evaluación será
hecha principalmente por médicos, es decir, será una evaluación ------.

\subsection{Research questions}

1. What is the best combination of filters?

\section{Objectives}
\subsection{General objective}

To create a data set with preprocessed mammograms with filters in the spatial
or spectral transform domain.

\subsection{Specific objectives}

To create an electronic file for each image.
To store the mammograms in a database oriented to storage and retrieval of images.
To create an online database of domain public.
To find what are the more suitable filters in the treatment of the image.

\section{Justification}

Our project is useful for the scientific community dedicated to processing and
image analysis, mainly those who are dedicated to the study of medical images.
As mention before our project could be a first step in creating a CAD system.
In the long term, the creation of this database would benefit the Mexican
radiologists. It will improve the state of the art medical image processing in
the country.

\section{Methodology}

We choosed a factorial design of experiments as a methodology.

x experimental and processing parameters were chosen to be analysed in this
research:

1. Variable A: Operator in a filter X 
2. Variable B: Operator in a filter X
3. Variable C: Operator in a filter X
Process of image preprocessing applied to our research:
1. Image digitalisation        
2. Image enhancing by filtering
3. Evaluation with both histogramas and medical opinion.
4. Presentation of the results

