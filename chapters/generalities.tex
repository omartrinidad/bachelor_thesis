\chapter{Introducción}
\label{generalidades}
% restart the count with arabic numbers
\pagenumbering{arabic} \setcounter{page}{1}

%\section{Antecedentes}

El cáncer de mama es un grave problema de salud pública, de acuerdo con la
Organización Mundial de la Salud en 2008 fue el tipo de cáncer más común entre
las mujeres y el que más muertes provocó~\cite{cancerreport}. Según el
INEGI las estadísticas son similares en México~\cite{mxcancer, inegi}. La
detección temprana de este padecimiento es la mejor forma de reducir el número
de decesos, a su vez, el método más comúnmente empleado para ello es el estudio
de mamografías\footnote{Los términos mamografía, mamograma y mastografía son
equivalentes y serán utilizados a lo largo de este documento indistintamente.}.

\begin{figure}[h]
    \centering

    \includegraphics[height=55mm]{images/art-mammogram.jpg}

  \caption[An image that can save a life] {An image that can save a life.
      Mamografía que muestra una lesión posiblemente cancerosa. Imagen
      presentada por los autores en la \nth{12} Conferencia Europea de Biología
      Computacional de la ISCB (International Society for Computational
  Biology) en la Exhibición de Arte y Ciencia.}

  \label{savealife}
\end{figure}

% tarea repetitiva, mejorar los de 'tarea subjetiva'
% peor aún --> even worse
Las mamografías son exámenes radiográficos diseñados para detectar el cáncer de
mama~\cite{bushberg2011essential}. El estudio de mamogramas es una tarea a
cargo de radiólogos, no es una actividad trivial, factores como la fatiga o el
uso de equipos de calidad no estándar influyen en la precisión del diagnóstico.
Dado que es una tarea subjetiva hay un margen de error en la opinión de los
radiólogos, lo que provoca diagnósticos conocidos como \textit{falsos
positivos} que son una falsa alarma al paciente o aún peor \textit{falsos
negativos}, que evitan que el paciente conozca la gravedad de su caso y se
someta a un tratamiento.

Los sistemas de \textit{detección asistida por computadora} (CAD) son
desarrollados con técnicas de reconocimiento de patrones para identificar
regiones sospechosas en una imagen y alertar de ello al radiólogo. Por otro
lado los sistemas de \textit{diagnóstico asistido por computadora} que al igual
que los primeros también son conocidos como CAD\footnote{Para distinguir entre
los sistemas de detección asistida por computadora y los sistemas de
diagnóstico asistido por computadora algunos autores suelen abreviar los
primeros como CADe y los últimos como CADi o CADx.} van más allá y su fin es
emitir un juicio sobre las lesiones, clasificando a estas como malignas o
benignas~\cite{castellino2005computer}.

Los sistemas CAD fueron diseñados para ayudar a los radiólogos en la
interpretación de imágenes médicas. Aunque la precisión de los sistemas CAD no
es perfecta~\cite{fenton2007influence} sirven como una segunda opinión para los
radiólogos. El desarrollo de los sistemas CAD ha sido vertiginoso desde los
años 80's~\cite{giger2008anniversary}.

% mandar lo de CPDM a pie de página ¿?
La elaboración de sistemas de este tipo es una tarea compleja, el primer reto a
sortear es obtener imágenes de prueba. El acceso a imágenes médicas con
propósitos de investigación es complicado debido a cuestiones de privacidad.
Este trabajo aborda la construcción de una colección de mamogramas digitales
preprocesados y de dominio público, bautizada como \textit{Colección de
Mamogramas Digitales Preprocesados} (CPDM, por sus siglas en inglés:
\textit{Collection of Preprocessed Digital Mammograms}). Los estudios fueron
recolectadas durante cuatro meses en Hospital de Alta Especialidad Dr. Juan
Graham Casasús, uno de los centros médicos más importantes del sureste del
país.

\section{Planteamiento del problema}

La concepción y creación de bases de datos\footnote{El contexto en el que se
utiliza el término base de datos en este trabajo en la mayoría de los casos no
se refiere a bases de datos relacionales.} de mamografías no es un tema nuevo.
Nishiwaka plantea un enunciado interesante: \textit{ En el futuro, las bases de
datos estarán constituidas por mamogramas digitales; actualmente (1998)
mamogramas digitalizados están siendo
utilizados~\cite{nishikawa1998mammographic}.} Las bases de datos mamográficas
juegan un rol importante en el desarrollo de algoritmos relacionados a los
sistemas CAD.

La concepción de una base de datos no es trivial, Moreira hace un recuento de
las características que una base de datos debe
tener~\cite{moreira2012compliance}. Las bases de datos deben cumplir ciertos
requerimientos que son resumidos en la Tabla~\ref{features}. Por su parte Zheng
establece que lo importante no es tener una gran cantidad de imágenes pero sí
una gran variedad~\cite{zheng2010computer}.

\begin{table}
  \caption[Requerimientos de una base de datos mamográfica]{Requerimientos de una base de datos mamográfica}
  \label{features}
\begin{center}
{\small
    \rowcolors{1}{}{lightgray}
    \begin{tabular}{c | >{\arraybackslash}m{3.5in}}
    \hline
    {\bf Requerimiento} &
    {\bf Descripción} \\
    \hline
    Selección de casos   & Lesiones de todo tipo y mamas con diferentes densidades\\
    Validez              & Los diágnosticos deben ser estrictamente corroborados
                           y donde sea posible exámenes de biopsia deben ser aplicados\\
    Información asociada & Historia clínica (edad, historial familiar, exámenes previos)\\
    Organización         & División adecuada de las imágenes, por casos, por paquetes\\
    Distribución         & Disponibilidad en internet\\
    Escáner (placas)     & El aparato para digitalizar debe ser de calidad\\
    \hline
    \end{tabular}
}
\end{center}
\end{table}

El problema que nosotros abordamos es la creación de una nueva colección de
mamogramas, cumpliendose la predicción de Nishikawa, la totalidad de los
mamografías que recolectamos son digitales. Nuestro enfoque incluye además la
peculiaridad de preprocesar cada ejemplar. El \textit{preprocesamiento} que es
la etapa previa al procesamiento de imágenes \textit{per se}, de
preparación~\cite{ponraj2011survey}.

\section{Objetivos}

\subsection{Objetivo general}

Crear un banco de datos de dominio público con mamografías digitales
preprocesadas.

\subsection{Objetivos específicos}

\begin{itemize}

    \item Revisar de forma exhaustiva la literatura existente sobre las
        técnicas de preprocesamiento de mamografías existentes y las colecciones
        de imágenes a las que actualmente se puede acceder.
    \item Preprocesar los mamogramas para eliminar el ruido y mejorar el
    contraste y también ejecutar otras tareas que resulten convenientes.
    \item Crear una ficha electrónica para cada caso que contenga datos
    relevantes como el diagnóstico médico.
\end{itemize}

\section{Justificación}

Aunque ya existen varias bases de datos, ninguna de ellas es una colección
mexicana y pocas de ellas son latinoamericanas. Por otra parte, la mayoría de
las bases de datos públicas que existen ofrecen mamografías tradicionales
\textit{digitalizadas} con escáners, una colección con imágenes totalmente
digitales supondría mejores resultados a la hora de ejecutar los algoritmos.

La utilidad de una colección de mamografías radica, en el corto plazo, en ser
un material para la comunidad científica dedicada al desarrollo de algoritmos
cuya fin es la construcción de sistemas CAD. También, de manera potencial y en
el largo plazo, en la disminución de muertes provocadas por el cáncer de mama.
Otro uso común es explotar las imágenes para la enseñanza y entrenamiento
de radiólogos principiantes.

% mejorar la redacción de este párrafo
Hay que señalar que la mayor parte de los sistemas CAD son comerciales y
representan un gasto para los hospitales. México es un país en vías de
desarrollo que adopta el papel de consumidor de tecnología. Las colecciones de
acceso público son recursos valiosos para la comunidad científica que trabaja en el
desarrollo de algoritmos para la elaboración de sistemas CAD y otras
tecnologías libres que permitan la reducción de la brecha tecnológica.

%\section{Caso de estudio}
%Poner aquí algo que sea relacionado a la metodología, revisión de literatura.

\section{Estructura de la tesis}

El documento está organizado como sigue: en el presente
Capítulo~\ref{generalidades} se da una revisión de las generalidades de nuestro
proyecto, después se detallan algunos conceptos sobre imágenes digitales y
también se hace una revisión de la literatura existente en el
Capítulo~\ref{marco}. En el Capítulo~\ref{desarrollo} se presenta el método
utilizado. En el Capítulo~\ref{resultados} se da cuenta de los resultados y
finalmente en el Capítulo~\ref{conclusiones} se hace un recuento de lo que se
hizo y se expone el trabajo futuro.
