\chapter{Generalidades}

% restart the count with arabic numbers
\pagenumbering{arabic} \setcounter{page}{1}

\section{Antecedentes}

El cáncer de mama es un grave problema de salud pública, de acuerdo con la
Organización Mundial de la Salud en 2008 el cáncer de mama fue el tipo de
cáncer más común entre las mujeres y el que más muertes provocó
\cite{cancerreport}. Las estadísticas son similares en Tabasco y México
\cite{inegi, mxcancer}. La detección temprana de este padecimiento es la mejor
forma de reducir el número de decesos, a su vez, el método más comúnmente
empleado en su detección es el estudio de mamografías, tarea a cargo de
radiólogos.

% tarea repetitiva, mejorar los de 'tarea subjetiva'
El estudio de mamogramas no es una actividad trivial, factores como la fatiga o
el uso de equipos de calidad no estándar influyen en la precisión del
diagnóstico.  Aunado a que es una tarea subjetiva y por lo tanto hay un margen
de error en la opinión de los radiólogos. Los sistemas de \textit{detección
asistida por computadora} o \textit{diagnóstico asistido por computadora} (CAD)
fueron desarrollados para ayudar a los radiólogos en la interpretación de
imágenes médicas. 

Se han hecho estudios sobre la precisión de los sistemas CAD.
\cite{fenton2007influence}.

La comunidad científica ha hecho un esfuerzo considerable para la detección
automatizada de lesiones mamográficas. 

El estudio de mamografías es un problema común en las ciencias computacionales,
sin embargo, es posible seguir contribuyendo a esta área del conocimiento.

Es díficil acceder a imágenes médicas con propósitos de investigación debido a
cuestiones de privacidad. Este trabajo aborda la construcción de una colección
de mamogramas preprocesados conocida como \textit{Collection of Preprocessed
Digital Mammograms} (CPDM).

Las imágenes fueron recolectadas durante cuatro meses en Hospital de Alta
Especialidad Dr. Juan Graham Casasús, que permitió la recolección de un lote de
mamogramas digitales.

\section{Planteamiento del problema}

Se dispone de un banco de imágenes mamográficas sin procesar, el reto es mejorar
su contraste, disminuir la cantidad de ruido.

Fue necesario descubrir que algoritmos son la mejor opción para mejorar la
calidad de la imagen.

Crear una ficha electrónica por cada estudio.

Los datos más relevantes son los diagnósticos BIRADS.

\section{Objetivos}

\subsection{Objetivo general}

Crear un banco de datos de dominio público con mamografías preprocesadas.

\subsection{Objetivos específicos}

Aplicar los algoritmos adecuados para preprocesar los mamogramas.
Crear una ficha electrónica para cada caso.

\section{Justificación}

La falta de una colección pública de mamogramas mexicanos.

El proyecto es útil para la comunidad científica dedicada al procesamiento y
análisis de imágenes médicas. A largo plazo, la creación de una colección de
este tipo beneficiaría a los radiólogos mexicanos.

Una colección de tal naturaleza también es útil para entrenar a radiólogos
novatos.
