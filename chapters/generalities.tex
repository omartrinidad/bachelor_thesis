\chapter{Generalidades}

% restart the count with arabic numbers
\pagenumbering{arabic} \setcounter{page}{1}

\section{Antecedentes}

El cáncer de mama es un grave problema de salud pública, es una de las
principales causas de decesos en la población femenina de Tabasco y México
\cite{inegi, mxcancer}. Una forma de aminorar estas muertes es la detección
temprana de este padecimiento, a su vez, el método más comúnmente empleado en
su detección es el estudio de mamografías, tarea a cargo de radiólogos.

El diagnóstico de mamogramas no es una actividad trivial debido a factores como
la fatiga, el uso de equipos de calidad no estandar, aunado a que es una tarea
subjetiva y por lo tanto hay un margen de error en la opinión de los
radiólogos. Es posible incrementar el éxito de los diagnósticos mejorando la
apariencia de la imagen, *introducir el preprocesamiento y escribir sobre los
sistemas CAD-

%Recordar escribir sobre (general purpose computers)

-El estado del arte del preprocesamiento de mamografías está avanzado-, sin embargo,
-es posible seguir contribuyendo a esta área del conocimiento-. 

Es díficil acceder a imágenes médicas con propósitos de investigación debido a
cuestiones de privacidad. Este trabajo aborda la construcción de una colección
de mamogramas preprocesados.

Dos bancos de datos similares al que se menciona en este trabajo son el Digital
Database for Screening Mammograpy (DDSM) de la University of South Florida
(USF) \cite{heath2000digital} y el banco de datos MIAS MiniMammographic
Database (mini-MIAS) \cite{sucklingmini}. Ambos son ampliamente usados por la
comunidad de investigación sobre procesamiento y análisis de imágenes. El
objetivo de este trabajo es crear un banco de datos similar a los que se
mencionan arriba. El proyecto fue -soportado- por el hospital Dr. Juan Graham
Casasús, que nos dió un lote de mamogramas en bruto.

especificar, el problema DEL
hospitlal --diagnóstico-- segunda opinión, decir que no se ha implementado algo
así, pero que existe
 
.... info on what to look at for image treatment 

La introducción de [15] da una explicación buena de por qué son buenos los
sistemas CAD.

\section{Problema}
\subsection{Planteamiento del problema}

Se dispone de un banco de imágenes mamográficas sin procesar, el reto es mejorar
su contraste, disminuir la cantidad de ruido.

Fue necesario descubrir que algoritmos son la mejor opción para mejorar la
calidad de la imagen.

Crear una ficha electrónica por cada estudio.

Los datos más relevantes son los diagnósticos BIRADS.

The assessment categories are:
\begin{enumerate}[a)]
    \item Mammographic assessment incomplete:
    \begin{enumerate}[i.]
        \item Categoría 0: Need additional imaging evaluation and/or prior
        mammograms for comparison.
    \end{enumerate}
    \item Mammographic assessment complete:
    \begin{enumerate}[i.]
        \item Categoría 1: Negative.
        \item Categoría 2: Benign finding(s).
        \item Categoría 3: Probably benign finding.
        \item Categoría 4: Suspicious abnormality.
        \item Categoría 5: Highly suggestive of malignancy.         
        \item Categoría 6: Proven malignancy.
    \end{enumerate}
\end{enumerate}

\subsection{Delimitación de la investigación}

decir que hay un estado inicial y uno final

la extensión del trabajo incluye --- filtros y tipos de filtros aplicados, imágenes listas para etapas posteriores. 
 --- estructura: imagen cruda, procesada y ficha

a database of preprocessed mammograms, namely, mammograms without noise,
reduction of artifacts and leveling of image quality.

\subsubsection{Limitations}

\subsection{Research questions}

1. What is the best combination of filters?

\section{Objectives}
\subsection{General objective}

Crear un banco de datos de dominio público con mamografías preprocesadas.

\subsection{Specific objectives}

Aplicar los algoritmos adecuados para preprocesar los mamogramas.
Crear una ficha electrónica para cada caso.

\section{Justification}

El proyecto es útil para

Our project is useful for the scientific community dedicated to processing and
image analysis, mainly those who are dedicated to the study of medical images.
As mention before our project could be a first step in creating a CAD system.
In the long term, the creation of this database would benefit the Mexican
radiologists. It will improve the state of the art medical image processing in
the country.

