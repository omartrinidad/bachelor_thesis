\newpage
\pdfbookmark[0]{Introducción}{Intro}
\chapter*{Introducción}

El uso de imágenes digitales es una actividad de suma importancia en la
medicina de hoy día. Uno de sus roles más importantes es la detección del
cáncer de mama. Gracias a sistemas de detección y/o diagnóstico asistido por
computadoras se puede mejorar la detección en etapas tempranas de esta
enfermedad.

Para agilizar la construcción de sistemas CAD es necesario contar con imágenes
de prueba, lo que no es trivial debido a cuestiones de privacidad. Sin embargo,
se vislumbra que las cosas cambien gracias a la incorporación de mamografías
digitales. En esta tesis se aborda la construcción de una colección de
mamografías digitales y el método de preprocesamiento que se les aplicó. Las
imágenes


