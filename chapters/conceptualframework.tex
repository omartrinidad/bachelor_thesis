\chapter{Theoretical framework}


\section{Frame of reference}
The pre-processing step serves to improve the processing with other filters
like “segmentation”, 

best techniques for enhancement of X-ray image may not be best for enhancement
for microscopic images. [6]

The enhancement methods can broadly be divided into the following two
categories: Spatial Domain Methods (SDM) and Frequency Domain Methods (FDM).
Spatial domain methods which are operate directly on pixels. Frequency domain
which operates on the Fourier transform of an image. [6]

Many studies show that the histogram equalization is an effective method for
improving the quality of medical images. [6, 8]. 

\subsection{Bith depths}

12-bit image (4096 levels of gray) displayed with 16 allocated bits appears
dark. This happens because the maximal possible amplitude of 12-bit image for
each pixel is 4095 and the maximal amplitude that can be displayed is 65535 (16
allocated bits). Therefore, 4096 levels need to be linearly scaled to 65536
levels to achieve good image displaying [8].


\section{Conceptual framework}

Digital image is a binary representation ...
Pixel
Histograms are frequency distributions, and histograms of images describe the frequency of the intensity values that occur in an image [9]. Image histogram offers a graphical representation of the tonal distribution of values in a digital image [10].
Color model
Grayscale
Image processing
Medical image processing
DICOM
***
Different kind of medical images
Las consideraciones técnicas (físicas) de cómo se obtienen los mamogramas están fuera del alcance de este trabajo. In screening mammography, as practiced in USA, (also in Mexico), two x-ray images of each breast, in the medio lateral oblique and craniocaudal views, are acquired. 
MRI
CT
X-ray:
Mammography is radiographic examination that is designed for detecting breast pathology, particularly breast cancer. [11]
***
Mammographic features characteristic of breast cancer are masses, particularly ones with irregular or “spiculated” margins; clusters of microcalcifications (tiny deposits of calcium); and architectural distortions of breast structures. 
***
Noise: 
Nuclear images are generally the most noisy. Noise is also significant in MRI, CT, and ultrasound imaging. In comparison to these, radiography produces images with the least noise. 
The noise can cover and reduce the visibility of certain features within the image. The loss of visibility is especially significant for low-contrast objects.
***


Noise in digital mammographies (Practical digital mammography by Beverly Hashimoto)
De acuerdo a Hachimoto en las imágenes mamográficas tenemos cuatro tipos de ruido:
*quantum
*fixed electronic
*señales secundarías
*quanta secundario indirecto

\subsection{Mamogramas}

Un mamograma típicamente tiene de 12 a 16 bits de profundidad, con una
resolución de 4000x5000 pixeles [13], [14].  

2.3. Technological framework


* Recordar escribir que en las primeras pruebas sólo se mejoró  la visualización del tejido graso y no del tejido mamario.


* Recordar escribir sobre (general purpose computers)


* Unsharp - masking to reduce noise.


Libros que ayudarán con los conceptos teóricos:


Formación física de las imágenes
Algoritmos
T
