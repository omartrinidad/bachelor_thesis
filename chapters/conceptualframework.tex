\chapter{Marco teórico}

\section{Imágenes médicas}
La unidad básica de una imagen es el pixel, imagen, escala de grises, procesamiento de
imágenes, procesamiento de imágenes médicas

\subsection{MRI, CT, X-ray}

\section{DICOM}
Mencionar DICOMDIR aquí.

\subsection{Estructura de un archivo DICOM}
Mencionar las etiquetas.

\subsection{Mamogramas}
A lo largo de este trabajo los términos mamografía, mastografía y mamograma
serán usados indistintamente. 

Un mamograma típicamente tiene de 12 a 16 bits de profundidad, con una
resolución de 4000x5000 pixeles [13], [14].  

Mammography is radiographic examination that is designed for detecting breast
pathology, particularly breast cancer. [11]

% --------------------
\subsection{Cáncer de mama} % buscar un mejor subtítulo
El cáncer de mama es un padecimiento en el que se desarrollan células malignas
en los tejidos de la mama. Nuevas celulas se forman cuando el cuerpo no las
necesita. 

\subsection{Proceso de formación de una mamografía}
Consideraciones técnicas avanzadas de cómo se obtiene un mamograma están fuera
del alcance de este trabajo. Obtener una imagen mamográfica es un desafío
debido a que la mama está constituida por tejidos similares entre sí y porque
las lesiones buscadas por el radiólogo que indicarían la presencia de un tumor
son pequeñas o similares al tejido normal \cite{mx:cancer}. Las mamografías son
imágenes obtenidas al exponer la mama a una dosis leve de rayos X. Los
mastógrafos disponen de un receptor que captura ...

\subsection{Preprocesamiento}
El preprocesamiento es la etapa previa al procesamiento de imágenes \textit{per
se}, el principal objetivo de esta etapa es mejorar la calidad de la imagen
para que quede lista para su posterior procesamiento \cite{ponraj2011survey}.

\subsubsection{Preprocesamiento de mamografías}
El preprocesamiento de mamografías es especial porque no es como los otros tipos
de preprocesamiento, hay preprocesamiento específico para mamogramas.

\section{Ruido}

\subsection{Ruido en mamografías}
Noise in digital mammographies (Practical digital mammography by Beverly
Hashimoto) De acuerdo a Hachimoto en las imágenes mamográficas tenemos cuatro
tipos de ruido:

\begin{enumerate}
    \item Ruido cuántico.
    \item fixed electronic
    \item señales secundarías
    \item quanta secundario indirecto
\end{enumerate}

\section{Histogramas}
Los histogramas son distribuciones de frecuencia, 

Histograms are frequency distributions, and histograms of images describe the
frequency of the intensity values that occur in an image [9]. Image histogram
offers a graphical representation of the tonal distribution of values in a
digital image [10].

% --------------------

Many studies show that the histogram equalization is an effective method for
improving the quality of medical images. [6, 8]. 

\section{Bits de profundidad}

12-bit image (4096 levels of gray) displayed with 16 allocated bits appears
dark. This happens because the maximal possible amplitude of 12-bit image for
each pixel is 4095 and the maximal amplitude that can be displayed is 65535 (16
allocated bits). Therefore, 4096 levels need to be linearly scaled to 65536
levels to achieve good image displaying [8].

\section{Conceptual framework}

In screening mammography, as practiced in
USA, (also in Mexico), two x-ray images of each breast, in the medio lateral
oblique and craniocaudal views, are acquired. 

Mammographic features characteristic of breast cancer are masses, particularly
ones with irregular or “spiculated” margins; clusters of microcalcifications
(tiny deposits of calcium); and architectural distortions of breast structures. 

* Recordar escribir que en las primeras pruebas sólo se mejoró  la
visualización del tejido graso y no del tejido mamario.

* Recordar escribir sobre (general purpose computers)

