\chapter{Conclusiones y trabajos futuros}
\label{conclusiones}

En este trabajo se abordó la creación de una colección de mamografías digitales
y el método de preprocesamiento que se les aplicó. Además como subproducto se
creó una interfaz gráfica que permite a los médicos manipular niveles de
contraste de la imagen. La colección es de dominio público. 

El método está formado de cinco fases, aunque no se implementó una evaluación
cuantitativa para comprobar la efectividad de los algoritmos empleados, un
grupo de médicos evaluaron los resultados de manera cualitativa, los resultados
son positivos.

El código fuente del método de preprocesamiento y también de la interfaz
gráfica está disponible en línea bajo el controlador de versiones Git en el
repositorio: \url{github.com/omartrinidad/preprocessing}. Se licenció el código
fuente con la licencia MIT\footnote{Massachussets Institute Technology} de
código abierto.

\section{Problemas abiertos}

Aunque CPDM es un recurso valioso, es necesario continuar el proceso de
recolección y también enriquecer las fichas de información incluyendo en estas
las coordenadas de las lesiones y también incluyendo la clasificación BIRADS en
los casos que no la tienen. En el nuevo proceso de recolección habría que tomar
en cuenta evitar la duplicación de los casos.

Tomando en cuenta que el sitio web actual está creado con archivos estáticos,
es posible construir un sistema web más robusto que permita al usuario hacer
consultas dinámicas. Es posible enriquecer el \textit{frontend} con un diseño
responsivo y el \textit{backend} con la inclusión de un sistema manejador de
base de datos. Por ejemplo, para DDSM, fue construido un \textit{servicio
web}~\cite{rose2006web}. 

Aunque la construcción de un sistema CAD es la contribución más destacada
derivada de una colección de este tipo, también se pueden generar sistemas de
entrenamiento para radiólogos noveles~\cite{suri2006recent} o incluso atlas de
lesiones mamográficas~\cite{antoniou2009web}.

% additions to the method
Es posible también abonar el método de preprocesamiento actual implementando
algoritmos de preprocesamiento orientado a mamografías como la remoción del
músculo pectoral en las imágenes con proyección MLO~\cite{raba2005breast}, la
detección del pezón~\cite{mendez1996automatic} o la orientación de la dirección
mama~\cite{masek2003automatic}.

Como se mencionó en el capítulo~\ref{marco}, no se implementó un algoritmo para
remover el ruido cuántico, por lo tanto es una tarea pendiente. Otra tarea
pendiente es implementar y/o modificar la basta cantidad de métodos existentes
sobre preprocesamiento de mamogramas, mejor aún, la colección es un recurso que
da pie a la creación de nuevos algoritmos.

% Rescaling. Pasar de una resolución a otra sin problemas.

