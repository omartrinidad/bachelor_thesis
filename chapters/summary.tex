\newpage 
\pdfbookmark[0]{Resumen}{Resumen}
\chapter*{Resumen}

La construcción de bases de datos mamográficas es una tarea vigente, sobre todo
con la reciente popularización de las mamografías digitales. El desarrollo de
sistemas CAD depende estrechamente de la existencia de colecciones de este
tipo. 

En esta tesis se expone la construcción de una colección de mamografías
digitales llamada Colección de Mamogramas Digitales Preprocesados. También se
aborda un método de prepocesamiento híbrido de cinco fases: reducción del área
de trabajo, conversión de bits, remoción de ruido utilizando el Filtro
Adaptativo de la Mediana, mejora de contraste a través de la ecualización de
histogramas y compresión utilizando el encogimiento de histogramas. Como
subproducto se aborda igualmente la construcción de una interfaz gráfica.

La evaluación de los resultados fue realizada por médicos del Hospital de Alta
Especialidad Juan Graham Casasús, de Tabasco, México. La colección de
mamografías está disponible públicamente en el siguiente sitio web:
\url{www.casi.dais.mx/cpdm/index.html}. 

\newpage 
\pdfbookmark[0]{Abstract}{Abstract}
\chapter*{Abstract}

The building of mammographic databases is a vigent work, more now with the
recent popularization of digital mammograms. The development of CAD systems
depends so much on the availability of collections of this kind.

In this thesis is exposed the building of a collection of digital mammograms
called Collection of Preprocessed Digital Mammograms. Also is addressed a
hybrid preprocessing method composed by five stages: reduction of the work
area, conversion of bits, denoising using the Adaptive Median Filter, contrast
enhancement through histogram equalization and compression using shrinking
histogram. As a subproduct equally is approached the building of a graphical
user interface.

The evaluation of results was realized by doctors from the Hospital de Alta
Especialidad Juan Graham Casasús, from Tabasco, Mexico. The collection of
mammograms is available publicly in the next website:
\url{www.casi.dais.mx/cpdm/index.html}. 
