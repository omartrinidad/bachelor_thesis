\chapter{Contribuciones}
\label{resultados}

En este capítulo se exponen las características de la colección de mamografías y
también se aborda una interfaz gráfica de usuario (GUI).

\section{Colección de mamografías}

La colección recibe el nombre \textit{Colección de Mamografías Digitales
Preprocesadas} (CPDM\footnote{Con el objetivo de que sea accesible a mayor
número de investigadores el nombre oficial de la colección está en inglés:
\textit{Collection of Preprocessed Digital Mammograms}.}. CPDM se constituye
como una fuente de imágenes médicas destinadas al uso de la comunidad
científica. Está disponible públicamente. La colección incluye las mamografías
crudas y las preprocesadas.

\section{Estructura de CPDM}

La Figura~\ref{fig:structure} nos muestra como están organizados los
directorios de la colección. Cada estudio tiene una ficha que contiene datos
del paciente como fecha de nacimiento, fecha del estudio, vista de la imagen,
nombre del archivo y el reporte BIRADS. Cada caso es almacenado en una carpeta,
que contiene la ficha descriptiva y las imágenes con extensión \texttt{.dcm}.
El nombre de cada imagen es la combinación del número de caso más la proyección
correspondiente, por ejemplo, la mama derecha con proyección MLO del primer
caso recibirá el nombre \texttt{0001rmlo.dcm}.

La colección está disponible en la siguiente dirección:
\url{www.casi.dais.mx/cpdm/index.html}. La evaluación de resultados se realizó
utilizando la opinión de los médicos.

\shorthandoff{>} % hack to combine tikZ and Spanish
    \begin{figure}[h]
\begin{center}
\
\tikzstyle{every node}=[draw=black,thick,anchor=west]
\tikzstyle{selected}=[draw=red,fill=red!30]
\tikzstyle{optional}=[dashed,fill=gray!50]

\begin{tikzpicture}[%
  grow via three points={one child at (0.5,-0.7) and
  two children at (0.5,-0.7) and (0.5,-1.4)},
  edge from parent path={(\tikzparentnode.south) |- (\tikzchildnode.west)}]
  \node {servidor}
    child { node [selected] {ftp}
        child { node {código fuente}}
        child { node {imágenes}
            child { node {crudas}
                child { node {caso 1}}
                child { node {...}}
                child { node {caso n}}
            }
            child [missing] {}				
            child [missing] {}				
            child [missing] {}				
            child { node {preprocesadas}
                child { node {caso 1}}
                child { node {...}}
                child { node {caso n}}
            }
        }
    }
    %child [missing] {}				
    child [missing] {}				
    child [missing] {}				
    child [missing] {}				
    child [missing] {}				
    child [missing] {}				
    child [missing] {}				
    child [missing] {}				
    child [missing] {}				
    child [missing] {}				
    child { node [selected] {http}
        child { node {archivos HTML}}
        child { node {archivos CSS}}
    };

\end{tikzpicture}

\end{center}

  \caption[Estructura de CPDM]{Estructura de CPDM}
  \label{fig:structure} 
\end{figure}

\shorthandon{>}

\section{Interfaz gráfica}
%colormap bone

Como subproducto del proyecto, construimos una interfaz gráfica de usuario
(GUI) siguiendo un enfoque similar a D'Angelo~\cite{d2007design}. Con la GUI
los médicos pueden modificar los parámetros de la función CLAHE. Aunque Matlab
ofrece el entorno GUIDE para crear interfaces escogimos crear la interfaz a
partir de cero debido a las flexibilidades de este enfoque.

La GUI desarrollada tiene un objeto \texttt{uipanel} a la izquierda y un
objecto \texttt{axes} a la derecha (ver Figura~\ref{main:before}). El objeto
\texttt{uipanel} contiene la vista previa de las imágenes, allí el usuario
puede elegir cualquiera de las imágenes con un click y en el objeto
\texttt{axes} se visualiza la elección del usuario.

\begin{figure}[h]
  \begin{center}
    {\includegraphics[height=40mm]{images/gui/main-before}}
  \end{center}
  \caption[GUI: Inicio]{Captura de pantalla de la GUI justo después de inicializar la aplicación}
  \label{main:before}
\end{figure}

Se agregó la opción de elegir uno o más archivos para que el usuario tenga la
posibilidad de escoger las diferentes proyecciones en cada caso médico (ver
Figura~\ref{fig:openfile}).

\begin{figure}[h]
  \begin{center}
    {\includegraphics[height=40mm]{images/gui/open-file}}
  \end{center}
  \caption[GUI: Selección de archivos]
  {Cuadro de diálogo que permite seleccionar una o más imágenes en formato DICOM}
  \label{fig:openfile}
\end{figure}

Se puede navegar en la imagen utilizando las teclas de movimiento o el juego de
teclas \texttt{hjkl} como en los sistemas operativos Unix. En la
Figura~\ref{main:after} se puede apreciar la vista previa de diferentes
proyecciones de una mama en el objeto \texttt{uipanel} y el acercamiento a la
imagen en el objeto \texttt{axes}.

\begin{figure}[h]
  \begin{center}
    {\includegraphics[height=40mm]{images/gui/main-after}}
  \end{center}

  \caption[GUI: Ventana principal 2]{Captura de pantalla de la GUI. A la
  izquierda el componente vertical es el objeto \texttt{uipanel} y a la derecha
  el componente \texttt{axes}}

  \label{main:after}
\end{figure}

La mayor parte de las operaciones de preprocesamiento ocurre al momento de
seleccionar y cargar las imágenes. Cuando la imagen está siendo visualizada ya
se ha reducido el área de trabajo, se ha realizado la conversión de bits y se
ha eliminado el ruido. El usuario tiene la posibilidad de mejorar el contraste
de la región visualizada en la sección derecha de la GUI. La interfaz cuenta
con una opción que permite modificar los parámetros de la función que
implementa el algoritmo CLAHE mencionado en la sección anterior (ver
Figura~\ref{options}).

Los parámetros modificables por el usuario son \texttt{Tiles by Column} y
\texttt{Tyles by Row}, elementos que establecen el número de divisiones
(mosaicos) de la imagen por columna y renglón respectivamente. \texttt{Contrast
Enhancement Limit}, que específica el valor límite de la mejora del contraste,
\texttt{Number of bins} es el número de columnas del histograma,
\texttt{Distribution} sirve para elegir la forma del histograma y con
\texttt{Range} se precisa el intervalo de los datos de la imagen de salida.

\begin{figure}[h]
  \begin{center}
    {\includegraphics[height=50mm]{images/gui/options}}
  \end{center}
  \caption[GUI: Modificación de parámetros]
  {Ventana que nos permite modificar los valores de entrada de la función CLAHE}
  \label{options}
\end{figure}

\section{Retroalimentación médica}

Con ayuda de médicos del Hospital Juan Graham Casasú se realizó una medición
cualitativa de los resultados. Se realizaron alrededor de cinco visitas
periódicas para obtener una retroalimentación de la mejora en la calidad de la
imagen. La calidad de la imagen es casi similar a la que se obtiene al utilizar
software comercial. Cabe destacar que en las primeras pruebas sólo se mejoró la
visualización del tejido graso y no del tejido mamario.


