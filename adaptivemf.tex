\pgfdeclarelayer{myback}
    \pgfsetlayers{myback,background,main}

\begin{figure}[h]
\centering

\subfloat[Ni $Zmed$ ni $Zxy$ son ruido\label{fig:nochanges}]{
    \begin{tikzpicture}

    \tikzset{myfillcolor/.style = {draw, fill=#1}}%
    \NewDocumentCommand{\fhighlight}{O{blue!25} m m}{%
        \draw[myfillcolor=#1] (#2.north west) rectangle (#3.south east);
    }
        \draw[xstep=0.80cm, ystep=0.80, color=blue] (0,0) grid (4,4);
        \matrix (m) [matrix of nodes,
        inner sep=0pt,
        anchor=south west,
        font=\sffamily\small,
        ampersand replacement=\&,
        nodes={inner sep=0pt, text width=0.80cm, align=center, minimum height=0.80cm}
        ]{
        159 \& 234 \& 210         \& 131 \& 120 \\
        392 \& 100 \& 91          \& 120 \& 138 \\
        205 \& 104 \& \textbf{107}\& 115 \& 233 \\
        332 \& 94  \& 96          \& 117 \& 420 \\
        39  \& 304 \& 349         \& 300 \& 130 \\
        };

    \begin{pgfonlayer}{myback}
    \fhighlight{m-2-2}{m-4-4}
    \fhighlight[green!40]{m-3-3}{m-3-3}
    \fhighlight[red!40]{m-3-2}{m-3-2}
    \end{pgfonlayer}

    \end{tikzpicture}
}
\hspace{12pt}
\subfloat[$Zxy$ es ruido\label{fig:zxynoise}]{
    \begin{tikzpicture} 

    \tikzset{myfillcolor/.style = {draw, fill=#1}}%
    \NewDocumentCommand{\fhighlight}{O{blue!25} m m}{%
        \draw[myfillcolor=#1] (#2.north west) rectangle (#3.south east);
    }

        \draw[xstep=0.80cm, ystep=0.80, color=blue] (0,0) grid (4,4);
        \matrix[matrix of nodes,
        inner sep=0pt,
        anchor=south west,
        font=\sffamily\small,
        ampersand replacement=\&,
        nodes={inner sep=0pt, text width=0.80cm, align=center, minimum height=0.80cm}
        ]{
        159 \& 234 \& 210         \& 131 \& 120 \\
        392 \& 99  \& 91          \& 135 \& 138 \\
        205 \& 109 \& \textbf{82} \& 117 \& 233 \\
        332 \& 90  \& 95          \& 149 \& 420 \\
        39  \& 304 \& 349         \& 300 \& 130 \\
        };

    \begin{pgfonlayer}{myback}
    \fhighlight{m-2-2}{m-4-4}
    \fhighlight[green!40]{m-3-3}{m-3-3}
    \fhighlight[red!40]{m-2-2}{m-2-2}
    \end{pgfonlayer}

    \end{tikzpicture}
}

% space
\subfloat[$Zmed$ es ruido\label{fig:zmednoise}]{
    \begin{tikzpicture}

    \tikzset{myfillcolor/.style = {draw, fill=#1}}%
    \NewDocumentCommand{\fhighlight}{O{blue!25} m m}{%
        \draw[myfillcolor=#1] (#2.north west) rectangle (#3.south east);
    }

        \draw[xstep=0.80cm, ystep=0.80, color=blue] (0,0) grid (4,4);
        \matrix[matrix of nodes,
        inner sep=0pt,
        anchor=south west,
        font=\sffamily\small,
        ampersand replacement=\&,
        nodes={inner sep=0pt, text width=0.80cm, align=center, minimum height=0.80cm}
        ]{
        159 \& 234 \& 110        \& 131 \& 120 \\
        192 \& 230 \& 230        \& 135 \& 138 \\
        105 \& 230 \& \textbf{3} \& 230 \& 233 \\
        132 \& 90  \& 230        \& 149 \& 220 \\
        139 \& 104 \& 149        \& 100 \& 130 \\
        };

    \begin{pgfonlayer}{myback}
    \fhighlight{m-2-2}{m-4-4}
    \fhighlight[green!40]{m-3-3}{m-3-3}
    \fhighlight[red!40]{m-2-2}{m-2-3}
    \fhighlight[red!40]{m-3-2}{m-3-2}
    \fhighlight[red!40]{m-3-2}{m-3-2}
    \fhighlight[red!40]{m-4-3}{m-4-3}
    \fhighlight[red!40]{m-3-4}{m-3-4}
    \end{pgfonlayer}

    \end{tikzpicture}
}
\hspace{12pt}
\subfloat[Incrementa el tamaño de $w$\label{fig:repeat}]{
    \begin{tikzpicture} 

    \tikzset{myfillcolor/.style = {draw, fill=#1}}%
    \NewDocumentCommand{\fhighlight}{O{blue!25} m m}{%
        \draw[myfillcolor=#1] (#2.north west) rectangle (#3.south east);
    }

        \draw[xstep=0.80cm, ystep=0.80, color=blue] (0,0) grid (4,4);
        \matrix[matrix of nodes,
        inner sep=0pt,
        anchor=south west,
        font=\sffamily\small,
        ampersand replacement=\&,
        nodes={inner sep=0pt, text width=0.80cm, align=center, minimum height=0.80cm}
        ]{
        159 \& 234 \& 110        \& 131 \& 120 \\
        192 \& 230 \& 230        \& 135 \& 138 \\
        105 \& 230 \& \textbf{3} \& 230 \& 233 \\
        132 \& 90  \& 230        \& 149 \& 220 \\
        139 \& 104 \& 149        \& 100 \& 130 \\
        };

    \begin{pgfonlayer}{myback}
    \fhighlight{m-1-1}{m-5-5}
    \fhighlight[green!40]{m-3-3}{m-3-3}
    \fhighlight[red!40]{m-5-1}{m-5-1}
    \end{pgfonlayer}

    \end{tikzpicture}
}
\caption[Funcionamiento del algoritmo AMF]{Funcionamiento del algoritmo AMF en diversos escenarios.}
\label{fig:amf} 
\end{figure}

