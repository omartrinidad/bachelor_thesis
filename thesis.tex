\documentclass[11pt, a4paper, oneside]{book}
\usepackage{tikz}
\usetikzlibrary{shapes, arrows, chains, positioning}
\usepackage[noend]{algpseudocode}
\usepackage{algorithm}
\usepackage[utf8x]{inputenc}
\usepackage[obeyspaces]{url}
\usepackage{graphicx,latexsym,amssymb,color}
\usepackage{subfig}
\usepackage{enumerate}
\usepackage{listings}

% configure links
\usepackage[]{hyperref}
\hypersetup{
    pdftex,
    colorlinks   = false,
    citecolor    = gray,
    pdfborder    = {0 0 0} % this solution is temporal
}

% changes in commmands
%\renewcommand{\rmdefault}{phv} % Arial
%\renewcommand{\sfdefault}{phv} % Arial

\begin{document}
\title{Creación de un banco de datos de mamogramas digitales}
\maketitle
\tableofcontents

\chapter{Introducción}
{\fontfamily{phv}\selectfont
Breast cancer is a major cause of death in Tabasco and Mexico. It is a second
cause of death for women aged 30-54. While this is a public health problem, a
way to decrease the number of deaths is by improving the detection of cancer.
The most common way to detect it is through the study of mammograms. This is
not an easy task, there is a margin of error in the opinion of the
radiologists.}

In turn, it is possible to increase the success of diagnoses
using CAD systems: Computer Aided-Detection (CADs) or Computer Aided-diagnosis
(CADx), that kind of systems assist doctors in the interpretation of
mammograms. CAD systems are built with artificial intelligence, digital image
processing, and pattern recognition.  An important step in building such
systems is the creation of databases of medical images. Most of the
mammographic databases are not publicly available, however, there are several
databases of this type. There are two databases of mammograms of public domain:
the USF (University of South Florida) database is a Digital Database for
Screening Mammography (DDSM) [1] and the MIAS MiniMammographic Database
(mini-MIAS) [2]. Both of them are widely used by the mammographic image
analysis research community.  Our main goal is to create a database similar to
the above.  Our project was supported by “Dr. Juan Graham Casasus” hospital who
gave us a batch of raw mammograms in DICOM file format [3].  especificar, el
problema DEL hospitlal --diagnóstico-- segunda opinión, decir que no se ha
implementado algo así, pero que existe
 

.... info on what to look at for image treatment 

La introducción de [15] da una explicación buena de por qué son buenos los
sistemas CAD.

\chapter{Resumen}
\chapter{Generalidades}
\section{Antecedentes}
\chapter{Marco teórico}
\chapter{Aplicación de la Metodología y Desarrollo}
\chapter{Pruebas y resultados}
\chapter{Conclusiones y trabajos futuros}

\end{document}
