
% -------------------------------------------------
% Set up a new layer for the debugging marks, and make sure it is on
% top

\pgfdeclarelayer{marx}
\pgfsetlayers{main,marx}
% A macro for marking coordinates (specific to the coordinate naming
% scheme used here). Swap the following 2 definitions to deactivate
% marks.
\providecommand{\cmark}[2][]{%
  \begin{pgfonlayer}{marx}
    \node [nmark] at (c#2#1) {#2};
  \end{pgfonlayer}{marx}
  } 
\providecommand{\cmark}[2][]{\relax} 
% -------------------------------------------------
\begin{figure}[h]
\begin{center}
\begin{tikzpicture}[%
    >=triangle 60,              % Nice arrows; your taste may be different
    start chain=going below,    % General flow is top-to-bottom
    node distance=6mm and 60mm, % Global setup of box spacing
    every join/.style={norm},   % Default linetype for connecting boxes
    ]

{\small\ttfamily\selectfont
%\fontfamily{lmodern}
% ------------------------------------------------- 
% A few box styles 
% <on chain> *and* <on grid> reduce the need for manual relative
% positioning of nodes
\tikzset{
  base/.style={draw, on chain, on grid, align=center, minimum height=4ex},
  proc/.style={base, rectangle, text width=8em},
  test/.style={base, diamond, aspect=2, text width=5em},
  term/.style={proc, rounded corners},
  % coord node style is used for placing corners of connecting lines
  coorda/.style={coordinate, on chain, on grid, node distance=6mm and 25mm},
  coordb/.style={coordinate, on chain, on grid, node distance=6mm and 35mm},
  coordc/.style={coordinate, on chain, on grid, node distance=6mm and 45mm},
  % nmark node style is used for coordinate debugging marks
  nmark/.style={draw, cyan, circle, font={\sffamily\bfseries}},
  % -------------------------------------------------
  % Connector line styles for different parts of the diagram
  norm/.style={->, draw, blue},
  free/.style={->, draw, green},
  cong/.style={->, draw, red},
  it/.style={font={\small\itshape}}
}
% -------------------------------------------------
% Start by placing the nodes
% Use join to connect a node to the previous one a

\node [term, it] (p0) {Start};
\node [proc, fill=violet!30, join] (p1) {Reducción del área de trabajo};
\node [test, fill=blue!30, join] (t1) {¿12 bpp?};
% No join for exits from test nodes - connections have more complex
% requirements
% We continue until all the blocks are positioned
\node [proc, fill=brown!30] (p2) {Conversión a 16 bits};
\node [proc, fill=red!30, join] (p3) {Filtro adaptativo de la mediana};
\node [proc, fill=green!30, join] (p4) {Ecualización de histograma};
\node [proc, fill=yellow!30, join] (p5) {Encogimiento de la imagen};
\node [term, join, it] (p6) {End};
% We position the next block explicitly as the first block in the
% second column.  The chain 'comes along with us'. The distance
% between columns has already been defined, so we don't need to
% specify it.

% -------------------------------------------------
% Now we place the coordinate nodes for the connectors with angles, or
% with annotations. We also mark them for debugging.

\node [coorda, left=of t1]  (c0)  {}; 
\node [coorda, right=of p3]  (c1)  {}; \cmark{1}   
\node [coordb, right=of p4]  (c2)  {}; \cmark{2}   
 
% -------------------------------------------------
% A couple of boxes have annotations

% -------------------------------------------------
% All the other connections come out of tests and need annotating
% First, the straight north-south connections. In each case, we first
% draw a path with a (consistently positioned) annotation node, then
% we draw the arrow itself.
\path (t1.south) to node [near start, xshift=1em] {$yes$} (p2);
  \draw [->,blue] (t1.south) -- (p2);

% -------------------------------------------------
% Finally, the twisty connectors. Again, we place the annotation
% first, then draw the connector
\path (t1.west) to node [near start, yshift=-3em, xshift=-1.3em] {$no$} (c0); 
  \draw [*->, blue, xshift=-1.3em] (t1.west) -- (c0) |- (p3);

\path (p3.east) to node [near start, yshift=1em] {} (c1); 
  \draw [o->, blue, dotted, thick] (p3.east) -- (c1) |- (p5);

\path (p4.east) to node [near start] {} (c2); 
  \draw [o->, red, dotted, thick] (p4.east) -- (c2) |- (p6);
% -------------------------------------------------

    %\draw[brown, thick] ($(p4.north west)+(-1.2, 0.3)$)  rectangle ($(p5.south east)+(1.2, -0.3)$) 
    %node [right, xshift=1em, yshift=5em]{GUI};
}
\end{tikzpicture}
\end{center}
  \caption[Método de preprocesamiento]
  {Método de preprocesamiento. \textbf{1} nos indica que es posible ir del paso}
  \label{fig:flowchart} 
\end{figure}
